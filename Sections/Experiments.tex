\documentclass[../Article_Model_Parameters.tex]{subfiles}
\graphicspath{{\subfix{../Figures/}}}
\begin{document}
	
	\label{CH: Experiments}
	
	In order to solve the optimization problem presented by Equation \ref{EQ: Optimization_formulation_MLE}, it was necessary to test the process model against the experimental dataset $Y(t)$, which was obtained by extracting oil from chamomile flowers. The experiments were performed by \citet{Povh2001} and \citet{Rahimi2011}, and conducted using a semi-batch extractor with a diameter of 3.96 cm and a length of 16.55 cm. Twelve experiments were performed under different operating conditions: $30-40^\circ C$, 100 - 200 bar and 0.12 - 0.24 kg/h. The amount of solid material used for extraction was 75 grams. The bulk density of the bed was computed based on the mass of the feed and the extractor volume. The bed porosity was determined using the following calculation: $\epsilon=1-\frac{d_a}{d_r} = 1-\frac{370}{1364} = 0.73$. It is assumed that the total amount of solute equals 3 grams, defined as the largest obtained yield after rounding up.
		
\end{document}