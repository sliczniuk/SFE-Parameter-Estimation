% ---------------------------------------------------------------
% Preamble
% ---------------------------------------------------------------
%\documentclass[a4paper,fleqn,longmktitle]{cas-sc}
\documentclass[a4paper,fleqn]{cas-dc}
%\documentclass[a4paper]{cas-dc}
%\documentclass[a4paper]{cas-sc}
% ---------------------------------------------------------------
% Make margins bigger to fit annotations. Use 1, 2 and 3. TO be removed later
%\paperwidth=\dimexpr \paperwidth + 6cm\relax
%\oddsidemargin=\dimexpr\oddsidemargin + 3cm\relax
%\evensidemargin=\dimexpr\evensidemargin + 3cm\relax
%\marginparwidth=\dimexpr \marginparwidth + 3cm\relax
% -------------------------------------------------------------------- 
% Packages
% --------------------------------------------------------------------
% Figure packages
\usepackage{graphicx}
\usepackage{float}
\restylefloat{table}
\usepackage{adjustbox}
% Text, input, formatting, and language-related packages
\usepackage[T1]{fontenc}
\usepackage{subcaption}

\usepackage{csvsimple}

% TODO package
\usepackage[bordercolor=gray!20,backgroundcolor=blue!10,linecolor=black,textsize=footnotesize,textwidth=1in]{todonotes}
\setlength{\marginparwidth}{1in}
% \usepackage[utf8]{inputenc}
% \usepackage[nomath]{lmodern}

% Margin and formatting specifications
%\usepackage[authoryear]{natbib}
\usepackage[sort]{natbib}
\setcitestyle{square,numbers}

 %\bibliographystyle{cas-model2-names}

\usepackage{setspace}
\usepackage{subfiles} % Best loaded last in the preamble

% \usepackage[authoryear,longnamesfirst]{natbib}

% Math packages
\usepackage{amsmath, amsthm, amssymb, amsfonts, bm, nccmath, mathdots, mathtools, bigints, ulem}

\usepackage{tikz}
\usepackage{pgfplots}
\usetikzlibrary{shapes.geometric,angles,quotes,calc}

\usepackage{placeins}

\usepackage[final]{pdfpages}

\usepackage{numprint}

% --------------------------------------------------------------------
% Packages Configurations
\usepackage{enumitem}
% --------------------------------------------------------------------
% (General) General configurations and fixes
\AtBeginDocument{\setlength{\FullWidth}{\textwidth}}	% Solves els-cas caption positioning issue
\setlength{\parindent}{20pt}
%\doublespacing
% --------------------------------------------------------------------
% Other Definitions
% --------------------------------------------------------------------
\graphicspath{{Figures/}}
% --------------------------------------------------------------------
% Environments
% --------------------------------------------------------------------
% ...

% --------------------------------------------------------------------
% Commands
% --------------------------------------------------------------------

% ==============================================================
% ========================== DOCUMENT ==========================
% ==============================================================
\begin{document} 
%  --------------------------------------------------------------------

% ===================================================
% METADATA
% ===================================================
\title[mode=title]{Supercritical fluid extraction of essential oil from chamomile flowers: modelling, and parameter estimation}                      
\shorttitle{Supercritical fluid extraction of essential oil from chamomile flowers: modelling, and parameter estimation}

\shortauthors{OS, PO}

\author[1]{Oliwer Sliczniuk}[orcid=0000-0003-2593-5956]
\ead{oliwer.sliczniuk@aalto.fi}
\cormark[1]
\credit{a}

\author[1]{Pekka Oinas}[orcid=0000-0002-0183-5558]
\credit{b}

\address[1]{Aalto University, School of Chemical Engineering, Espoo, 02150, Finland}
%\address[2]{2}

\cortext[cor1]{Corresponding author}

% ===================================================
% ABSTRACT
% ===================================================
\begin{abstract}
This study investigates the supercritical extraction process of essential oil from chamomile flowers. Essential oils of chamomile are used extensively for medicinal purposes. Many different chamomile products have been developed, the most popular of which is herbal tea. In this study, a mathematical model describing the governing mass transfer phenomena in solid-fluid environment at supercritical conditions using carbon dioxide is formulated. The concept of quasi-one-dimensional flow is applied to reduce the number of spatial dimensions. The flow of carbon dioxide is assumed to be uniform across any cross-section, although the area available for the fluid phase can vary along the extractor. The physical properties of the solvent are estimated based on the Peng-Robinson equation of state. Model parameters, including the partition factor, internal diffusion coefficient, and decaying factor, were determined through maximum likelihood estimation based on experimental data assuming normally distributed errors. A set of laboratory experiments was performed under multiple constant operating conditions: $30 - 40^\circ C$, $100 - 200$ bar, and $3.33-6.67 \cdot 10^{-5}$ kg/s.

%A distributed-parameter model describes the fluid-solid extraction process.

\end{abstract}

\begin{keywords}
Supercritical extraction \sep Parameter estimation \sep Mathematical modelling
\end{keywords}

% ===================================================
% TITLE
% ===================================================
\maketitle

% ===================================================
% Section: Introduction
% ===================================================\section{Introduction}

\section{Introduction}
\subfile{Sections/introduction_imp}

\subfile{Sections/Model}

\subsection{Parameter estimation} \label{CH: Parameter_estimation}
\subfile{Sections/Parameter_estimation}

\subsection{Experimental work}
\subfile{Sections/Experiments}

\section{Results}
\subfile{Sections/Results_Chamomile}

\section{Conclusions} \label{CH: Conclusion}

The article presents a comprehensive study on supercritical fluid extraction of essential oil from chamomile flowers, focusing on developing and applying a distributed-parameter model to describe the fluid-solid extraction process. By employing the concept of quasi-one-dimensional flow, the study simplifies the spatial dimensions of the extraction process, ensuring uniform flow across any cross-section while allowing for variations in the area available for the fluid phase. The physical properties of the solvent are estimated using the Peng-Robinson equation of state.

The model calibration is based on the laboratory experiments data conducted by \citet{Povh2001} under various conditions. The model parameters, such as partition factor, internal diffusion coefficient, and decaying factor, were determined through maximum likelihood estimation based on experimental data. The parameter space exploration revealed that while some parameters could be determined with a high degree of confidence, others, like the partition factor, had a low impact on the model's output. The identification of low-impact parameters leads to model reduction. This work introduced a set of correlations to find a general relationship between parameters and the operating conditions. The obtained correlations were incorporated into the process model and tested against the dataset. The process model is capable of reproducing the dataset with satisfactory accuracy.

The process model can be further used with the presented correlations to introduce an extraction model with dynamically changing operating conditions for multiple purposes, such as yield maximisation, local sensitivity analysis, techno-economic analysis, or optimal experiments design.

% ===================================================
% Bibliography
% ===================================================
%% Loading bibliography style file
%\clearpage
\newpage
%\bibliographystyle{model1-num-names}
\bibliographystyle{unsrtnat}
\bibliography{mybibfile}

\clearpage \appendix \label{appendix}
\section{Appendix} 

%\subsection{Thermodynamic}
\subfile{Sections/Qubic_EOS} \label{CH: EOS}

\subsection{Cardano's Formula} \label{CH: Cardano}
\subfile{Sections/Cardano}

\end{document}